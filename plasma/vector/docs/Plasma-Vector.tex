\documentclass[11pt]{article}
\usepackage[margin=1.5in]{geometry}
\title{Plasma Vector [WIP]}
\author{Nathan Ginnever}
\date{\today}

\begin{document}
\maketitle
\centerline{\textit{Introduction}}

This work presents an application of cryptographic universal accumulators [5] that was expanded upon in the blockchain setting by Bunnz et al [1]. We apply these RSA accumulators in Plasma [2] and show that Vector can scale to high transaction throughput with low overhead for clients (potentially at the expensive of increased prover complexity from the operator or service nodes...TODO). Vectors [] (VC) have the property of binding data to a position in the list and we explore why this is a desired property of accumulators in the Plasma application setting. Using VC with 160 bit message spaces, we show how to construct a key-value map in an accumulator and its application to Cash [] history proofs. Our contribution is applying VCs to reducing the inclusion and exclusion proof size of Plasma Cash/flow [] from an $O(n)$ and $O(b \log n)$ respectively, where $b$ is a number of blocks committed and $n$ is the number of coins a user owns, to an $O(1)$ for both. We will see that replacing the merkle tree entirely with a VC accumulator can increase the granularity of Plasma Cash/flow to allow for smaller payments and \textit{fragmentation} no longer is an issue.

As with previous Plasma designs, this paper will detail the \textit{deposit}, \textit{send}, and \textit{exit} components which follow directly from Cash[]. The Plasma Vector parent contract does not verify the correctness of state transitions, it only provides an immutable ledger to record accumulator commitments. We fall back on the Plasma Cash/flow challenge functions to ensure that invalid transitions cannot exit. In this case, the state transition is simply defined as updating the owner public key of a position(s) in the accumulator. Verifying the correctness of accumulator updates involves inspecting the public key at position $x_i$ in $A$, the previous committed accumulator value, matches the signature witness and that $A'$ updates the value at $x_i$ to the new public key. 
\\
\\

\centerline{\textit{Vector Commitments}}

A \textit{vector commitment} (VC) [] is a special case of RSA accumulator with the added property of \textit{position binding}. RSA accumulators prove inclusion and exclusion of prime numbers in a set, but do not allow for arbitrary values to be accumulated to preserve the security properties of the RSA group. VC allow for an $M$-bit $\{0,1\}^\lambda$ message to be committed to each of the accumulator vector components by assigning prime numbers to the message bits. Committing to a vector allows the accumulator to act like a sparse merkle tree where we can make a statement that for $x$, a given vector, that for any component index $x_i$ opening a value $v$ that a forger cannot provide a $v'$ s.t. $x_i$ opens to both $v$ and $v'$. While a merkle tree has a proof size of $O(\lambda \log n)$, where $\lambda$ is the security parameter of the hash function, to open a value at a position, VC have just constant sized openings. They also carry over the batching properties derived from previous accumulator schemes [ref accumulators] so that Plasma Vector can batch proofs that any number of values are committed to a given index in constant space.
\\
\\

\centerline{\textbf{VC Accumulator Construction}}

We first construct a classic RSA accumulator as described in [CL02 Lip12]. Choose a group $G$ in ${Z/ZN}^*$ where $N=pq$ and $|N| = (p-1)(q-1)$ is unknown. $pq$ are two 1024 bit prime numbers $(p,q)$. Select a generator $g \in G$ and let the CRS be (N, g).
\\

\centerline{\textit{Setup}}

For a VC we add three more properties to the setup procedure, the size of the vector $n$ (note this is smaller than requiring the set $x$ for all $x_1...x_n$), a function map to primes $H_{prime}$, and a message space $M$ (size of each component message $x_i$ in the vector). For Plasma cash, $n$ will be set to the number of lowest denomination coins we would like the vector to represent. I.e. if we would like to have a plasma chain that can \textit{fragment} [cash doc] to a total of $2^{40}$ owned ranges we would set $n=2^{40}$ and $H_{prime}$ will need to map $[0,n]$ to a unique prime.
\\

\centerline{\textit{Update}}

We define now how to update our VC. Let’s start with a trivial VC example and assume that the message space of $M = \{0,1\}$ for simplicity, but we can extend this space to hold enough bits to register a public ethereum key in the accumulator. This will allow the operator to change the ownership of a coin ID without needing to commit to a seperate merkle tree to include this data. Now for a set of $m_i$ messages to be added to $A$ we generate $p_i = H_{prime}(m_i)$ for all $m_i$ != 0. We can define $A=g^{init}$ and generate an update as $A’ = A^{\prod_{i=1, m_i=1}^{n}p_i}$. I.e. $g=3$, $A_0=g^{3^0*5^0*7^0}$ Here we have an accumulator that is set up to hold three coins and is empty as they are all raised to the zero power. Since our message space is only \{0,1\}, we can only commit the value 1, so to commit a value to the first and third position, we can accumulate 3 and 7 into $A$ as follows.  $A_1 = A_0^{3^1*5^0*7^1}$.  Now $A_1$ contains the value 1 at the first position $x_1$ in our accumulator, $x_2$ contains 0 and $x_3$ contains 1 to build a vector of 1-bits [1,0,1].
\\

\centerline{\textit{Open and Verify}}

Here, in our example, the verifier can ask the operator to open the committed vector at position 1 that outputs proof $\pi$ that can verify that the value is 1 at position 1. To do this we generate a membership witness $w$ for $p_i$ in $A_1$. Proving that $m_i$ is set to 0 involves generating a \textit{exclusion} proof however. Note that when a VC only contains 1-bit messages it takes on the form of classic RSA [] accumulator. I.e. Ask to open position 1. Since $x_1 = 1$ the operator will generate an RSA [] inclusion proof $w= 3*7/7 = 3$ and perform a [wes18] PoKE for large cofactors (omitted here). The verifier will compute that the $g^{w*3} = A_1$ 
\\

\centerline{\textit{Extending to Arbitrary Message Space}}

Here we increase the amount of data that our vector components can hold to that of 160 bits for an ethereum public key. To do this, we notice that we now need a unique prime for each bit committed. Formally we need to modify $H_{prime}$ to output $\lambda$ primes for a given $i$ input. Consider the message space $\{0,1\}^{\lambda}$ with hash function $H: M \rightarrow \{0,1\}^{\lambda}$. $H_{prime}$ should associate primes $\{p_j: j \in [i\lambda,(i+1)\lambda)$. In our case, $\lambda = 160$ bits. Given a coin vector space of dimension $2^{40}$, we would require $160*2^{40}$ which is approximately $2^{48}$ unique primes and still reasonable to generate in the setup phase. On average we will need 80 inclusion proofs and 80 exclusion proofs per coin index opening. This is a pain point as a public key may have many 0 bits, and we cannot batch exclusion proofs. We consider the idea of implementing two accumulators, one where accumulating a prime represents 0, and the other represents 1. This way we can reduce to the problem of proving commitment to a public key to only inclusion proofs. Given that each inclusion proof is ~2048 bits we can derive a proof that all coins are owned by a user in 4096 bits. [todo… verify these numbers, don’t think they are correct] I.e.

Have the operator commit to two VC accumulator values $A_i$ and $A_e$ where including a prime in $A_i$ represents committing to the inclusion of a value i.e. a bit is 1 not a 0. And then $A_e$ represents committing to exclusion of a value i.e. a bit is 0 not 1 at a given position $i$ in vector $x$.
In Plasma Vector we want a sparse vector that stores only the public key of the owner of the coin at a particular index. Assume a public key is only 8 bits i.e if we have two values $v_1 = [01101110]$ and $v_2 = [10100111]$ we could generate the two VC accumulators $A_i$ and $A_e$ like so…
\\
\\
Let $v_1^p = [3,5,7,11,13,17,19,23]$
\\
Let $v_2^p = [29,31,37,41,43,47,53, 59]$
\\
Let $A_i = g^{[{3^0}*{5^1}*{7^1}*{11^0}*{13^1}*{17^1}*{19^1}*{23^0}]*[{29^1}*{31^0}*{37^1}*{41^0}*{43^0}*{47^1}*{53^1}*59^1]}$
\\
Let $A_e = g^{[{3^1}*{5^0}*{7^0}*{11^1}*{13^0}*{17^0}*{19^0}*{23^1}]*[{29^0}*{31^1}*{37^0}*{41^1}*{43^1}*{47^0}*{53^0}*59^0]}$
\\

To prove that $x_1$ opens to $v_1$ we would ask for a batched inclusion proof for 5,7,13,17,19 to prove that $x_1$ position contains all of the correct bits with 1. To prove $x_1$ has all of the correct 0 bits committed we would ask for a batched inclusion proof for 3,11,23 which should result in just two times the batched proof size of one accumulator. Not explicitly proving exclusion could mean that multiple values are stored in the same position since the operator could include more primes than the verifier is checking which will allow other values to pass. To fix this we use [] to compactly prove that $A_i$ and $A_e$ are disjoint accumulator sets.
\\
\\

\centerline{\textbf{Proof of Exponent Knowledge \textbf{PoKE}}}

[todo]

Here we define the protocols that Plasma Vector uses to ensure that the inclusion and exclusion proof size is manageable. Recall [vitalik post], that for a given RSA accumulator $A = g^u$, if we would like to show that an element $v$ is in $A$ the prover must generate a cofactor $x$ s.t. $A = (g^v)^x$. In our case $x$ will be large as it represents all IDs in our coin space, so using it as part of the proof is not feasible. Using an extension of the Wesoloski PoE [] proposed by [bunnz] reduces the burden of passing $x$ which is approx $|u|$ we now only pass values $|N|$ and $B$ which are only 1024 bit security parameters.



\\ 
\\
\centerline{\textbf{Hash to Prime}}

There are a few methods for assigning prime numbers and two places in Plasma vector that this will be required. In the PoKE scheme detailed above, $B$ is the output of a $H_{prime}$ function, and all of the indices of our sparse VC require an output from a $H_{prime}$. In order to appropriately choose our output spaces, we need to observe the distribution of primes in the number line. Luckily prime numbers occur quite frequently. To illustrate this we can roughly estimate the density of primes using $n/\ln(n)$. Given $10^{50}$ numbers we might expect $10^{50}/\ln(10^{50})=10^{40} $ primes.

A strategy for $H_{prime}$ for PoKE is using an iterative function and a counter to hash from the domain $Z_{2^{\lambda}}$ to prime [CS99][FT14]. We can use a function that assigns smaller primes to the vector indices as was described by [] with, $f(i)=2(i+2)* \log_2(i+2)^2$. If primality checking becomes too expensive then we may assign arbitrary large integers that have prime factorizations[TODO check this]. Additionally the prover could provide a nonce s.t. $H(nonce||index)=l$ with $l \in Primes(\lambda)$. This would reduce the primality checks to just one instead of $\lambda$ as is with the iterative counter function method.

A security requirement of the Wesoloski PoKE is that $B > n$ where $n$ is the length of the vector $A$. If $n=2^{48}$ primes then we will have to ensure that $B > 10^{15}$ and prime. We can make a criteria of the prime selection choose primes within the range of $[2^{48}, 2^{62}]$.
\\
\\
\centerline{\textbf{Deposit}}

This is done in the same way as Plasma Cash with an onchain token deposit to the plasma parent contract. The only difference here will be the assignment of a unique vector that will identify a new range of tokens. We have $2^{48}$ primes to represent 160-bit public keys in a vector of size $2^{40}$. With 
\\
\\
\centerline{\textbf{Send}}

The send format allows for an owner of a range of coins to be able to generate a transaction that assigns all or part of their range to a new owner. 
\\
\\
\centerline{\textbf{Exit}}

Since we have no merge protocol, we must exit all fragmented ranges separately. Atomic swaps may help to defragment ranges. [todo expand]
\\
\\
\centerline{\textbf{Smart Contract}}

[TODO]
\\
\\
\centerline{\textbf{References}}

1 \url{}

2 \url{}

3 \url{}

4 \url{}

5 \url{}



\end{document}
